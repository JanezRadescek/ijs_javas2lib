\documentclass[english]{article}
\usepackage[english]{babel}
\usepackage[utf8]{inputenc}
\usepackage[T1]{fontenc}
\usepackage{listings}

\begin{document}

\section{Overview}

This article provides some information about Cli alongside with examples on how to use it.

\section{Info}

Cli provide tools to some simple S2 file manipulation. In its current form it can generate basic information about S2 file, generate CSV file from data in S2 file, select desirable part of S2 file and merge two S2 files into one s2 file.
We control program with flags and their arguments.

\subsection{Flags}
Whenever we run Cli there are two mandatory flags, task flag and input flag, followed by additional flags. Only one task flag can be used at the same time. Flags arguments must follow directly after flag and in correct order.


\subsubsection{statistics}
\textbf{-s} provides us with some statistics. It is task flag. It has no arguments. By default it will write result to standard output. If we provide \textbf{-o} it will write result to this file instead. We will get  version; total time of measurement; date, time and zone of the beginning; no.\ of special messages, comments,   definitions, time stamps, unknown, errors, streams,packets per stream,samples per stream.

\subsubsection{CSV}
\textbf{-r} partially convert S2 file into CSV format(it is irreversible, as we keep only data and their respective times). It is task flag. Has no arguments. By default it will 
\begin{itemize}
\item  write result to standard output,
\item use all handles,
\item start from time 0 and end on Long.MAXVALUE = 9,223,372,036,854,775,807.
\end{itemize}

If we provide \textbf{-o} it will write result to this file instead.
If we provide \textbf{-h} it will only use those handles.
If we provide \textbf{-t} it will only use data from that time interval.

We will get time stamps, data, and handles on the output.

\subsubsection{Cut}
 \textbf{-c} cut original S2 file and save new one. It is task flag. Has no arguments. It is mandatory to accompany this flag with flag \textbf{-o}.
By default it will copy original meaning:
\begin{itemize}
\item use all handles,
\item start from time 0 and end on Long.MAXVALUE = 9,223,372,036,854,775,807.
\item keep comments, special messages,...
\end{itemize}

If we provide \textbf{-h} it will only use those handles.
If we provide \textbf{-t} it will only use data from that time interval.
If we provide \textbf{-d} it will only use those additional data types.

\subsubsection{Merge}
 \textbf{-m} merges two original S2 files and save them into new one. It is task flag. It has mandatory argument. If argument is true it check if files correspond and if they do it will merge them on same handles, do nothing otherwise. If argument is false it will give second file new handles. It is mandatory to accompany this flag with flag \textbf{-o}.


\subsubsection{Input}

 \textbf{-i} is input flag. It has 1+1 arguments. First argument is file path of main input S2 file. Second argument is optional. It is file path for secondary input file and is used only in combination with flag \textbf{-m}.

\subsubsection{Output}
\textbf{-o} is additional flag. It has 1 mandatory argument. Argument is file path of output file. In case file with same file path already exist it will be overwritten.
When used with
\begin{itemize}
\item \textbf{-s} file extension must be \textbf{.txt}
\item \textbf{-r} file extension must be \textbf{.csv}
\item \textbf{-c} file extension must be \textbf{.s2}
\item \textbf{-m} file extension must be \textbf{.s2}
\end{itemize}

\subsubsection{Handles}
 \textbf{-h} is additional flag.  It is used in combination with \textbf{-r} or \textbf{-c}. It has mandatory argument. Argument is sequence of zeros and ones. If sequence contains 1 on position \textbf{i+1} from right to left it will process handle \textbf{i}. 0 on position \textbf{i+1} from right to left means this handle will not be processed. 0 left to the leftmost 1 can be omitted. Keep in mind current version of S2 supports only handles 0-31. If this flag is omitted  completely it is equivalent to using this flag with argument of sequence of 32 ones.

\subsubsection{Time interval}
\textbf{-t} is additional flag. It is used in combination with \textbf{-r} or \textbf{-c}. It has 2 mandatory arguments and one optional. First argument must be smaller than second one. The arguments represent time interval on witch data should be processed. If the third arguments is \textbf{true} it will approximate comments and special messages with last explicit time and process them accordingly. Otherwise comments and special messages will be processed as they are all on time interval. 

\subsubsection{Data types}
\textbf{-d} is additional flag. It is used in combination with \textbf{-c}.  It has mandatory argument. Argument is sequence of zeros and ones. 1 on position $i$ from right to left means it will keep:
 \begin{itemize}
\item i=1 comments
\item i=2 special messages
\item i=3 metadata
\end{itemize}
Current version needs metadata for correct merging therefore its strongly recommended to never delete those.
Omitting this flag is equivalent to \textbf{-d 111}.

\section{Examples}

In all examples to follow we will assume we have two \textbf{S2} files named \textbf{file1.s2} and \textbf{file2.s2} both stored in \textbf{./files/} . Examples are independent.


\subsection{example}
Lets say file1.s2 stores data about EKG measurement and we want to know how long did it last. For that we call Cli as follows. $$\textbf{Cli -s -i ./files/file1.s2} $$
 \textbf{-i} is always necessary and has 2 mandatory arguments file directory and name. After running the program we will get statistics of \textbf{file1.s2} on standard output.

Now we want to know the actual data for the first 30s. We want them saved in file \textbf{output1.csv} for later use: $$ \textbf{Cli -r -i ./files/file1.s2 -o ./files/output1.csv -t 0 30} $$
Flag \textbf{-r} tels the program we want actual data in CSV format, \textbf{-o} has 2 necessary arguments directory and name of file in which we will save our CSV data. There is also \textbf{-t} with 2 arguments which represent time interval.


\subsection{example}
Lets say the measurement on \textbf{file2.s2} is too long. We decide we only want part of data between 45s and 75s since the beginning. We also dont want to keep special messages.We call Cli as follows.
$$ \textbf{Cli -c -i ./files/file2.s2 -o ./files/newFile2.s2 -t 45 75 -d 101}$$
Now we want data from \textbf{file1.s2} and \textbf{newFile2.s2} to be in the same S2 file named \textbf{merged.s2}
$$ \textbf{Cli -m false -i ./files/file1.s2 ./files/newFile2.s2 -o ./files/merged.s2}$$


\subsection{example}
Let say we have 2 mesurments saved on \textbf{file1.s2}. We are particularly interested in data between 25-30 seconds and 130-205 seconds in first mesurement. First we cut each part out and save it. 
$$ \textbf{Cli -c -i ./files/file1.s2 -o ./files/cut1.s2 -t 25 30 -h 0}    $$
$$ \textbf{Cli -c -i ./files/file1.s2 -o ./files cut2.s2 -t 130 205 -h 0}$$
Now we merge them back into new file. Since they have data from same initial measurement we want them to look like it.
$$ \textbf{Cli -m true -i ./files/cut1.s2 ./files/cut2.s2 -o ./files/merged.s2}$$


\end{document}